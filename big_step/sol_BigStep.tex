%&PDFLaTeX
\documentclass[a4paper, 11pt, ngerman]{scrartcl}
\usepackage[utf8]{inputenc}
\usepackage[ngerman]{babel}
\usepackage[T1]{fontenc}
\usepackage{hyphenat}
\usepackage{mathpartir}

\usepackage{geometry}
\geometry{inner=2.0cm,outer=2cm,top=2.5cm,bottom=2.5cm,head=28pt}

\setlength{\parskip}{2em}

\begin{document}

\small

(*0*)
\[
\inferrule* [left=Op]
  {4 \Rightarrow 4 \\ 5 \Rightarrow 5 \\ 4+5 \Rightarrow 9}
  {4+5 \Rightarrow 9}
\]

Axiome $v \Rightarrow v$ werde ich in ab hier meistens weglassen


\[
\inferrule* [left=App]
  {
		\inferrule* [left=GD]
			{f = fun\ x \rightarrow 2*x}
			{f \Rightarrow fun\ x \rightarrow 2*x}
		\\
		\inferrule* [left=Op]
			{2*4 \Rightarrow 8}
			{2*4 \Rightarrow 8}
	}
  {f\ 4\Rightarrow 8}
\]


(*1*)
\[
\inferrule* [left=App]
  {
		\inferrule* [left=GD]
			{f = fun\ x \rightarrow 2*x}
			{f \Rightarrow fun\ x \rightarrow 2*x}
		\\
		\inferrule* [left=App]
			{
				\inferrule* [left=GD]
					{g = fun\ x \rightarrow x-1}
					{g \Rightarrow fun\ x \rightarrow x-1}
				\\
				\inferrule* [left=Op]
					{2-1 \Rightarrow 1}
					{2-1 \Rightarrow 1}
			}
			{g\ 2\Rightarrow 1}
		\\
		\inferrule* [left=Op]
			{2*1 \Rightarrow 2}
			{2*1 \Rightarrow 2}
	}
  {f\ (g\ 2)\Rightarrow 2}
\]

\[
\inferrule* [left=App]
  {
		\inferrule* [left=App]
			{
				\inferrule* [left=GD]
					{diff = fun\ x \rightarrow fun\ y \rightarrow x-y}
					{diff \Rightarrow fun\ x \rightarrow fun\ y \rightarrow x-y}
			}
			{diff\ 3\Rightarrow fun\ y \rightarrow 3-y}
		\\
		\inferrule* [left=Op]
			{3-2 \Rightarrow 1}
			{3-2 \Rightarrow 1}
	}
  {diff\ 3\ 2\Rightarrow 1}
\]


(*2*)
\tiny
\[
\inferrule* [left=App]
  {
		\inferrule* [left=App]
			{
				\inferrule* [left=GD]
					{diff = fun\ x \rightarrow fun\ y \rightarrow x-y}
					{diff \Rightarrow fun\ x \rightarrow fun\ y \rightarrow x-y}
				\\
				\inferrule* [left=App]
					{
						\inferrule* [left=GD]
							{f = fun\ x \rightarrow 2*x}
							{f \Rightarrow fun\ x \rightarrow 2*x}
						\\
						\inferrule* [left=Op]
							{2*5 \Rightarrow 10}
							{2*5 \Rightarrow 10}
					}
					{f\ 5\Rightarrow 10}
			}
			{diff\ (f\ 5)\Rightarrow fun\ y \rightarrow 10-y}
		\\
		\inferrule* [left=Op]
			{10-5 \Rightarrow 5}
			{10-5 \Rightarrow 5}
	}
  {diff\ (f\ 5)\ 5\Rightarrow 5}
\]

\[
\inferrule* [left=LD]
  {
		5 \Rightarrow 5 \\
		\inferrule* [left=App]
			{
				\inferrule* [left=GD]
					{g = fun\ x \rightarrow x-1}
					{g \Rightarrow fun\ x \rightarrow x-1}
				\\
				\inferrule* [left=App]
					{
						\inferrule* [left=GD]
							{f = fun\ x \rightarrow 2*x}
							{f \Rightarrow fun\ x \rightarrow 2*x}
						\\
						\inferrule* [left=Op]
							{2*5 \Rightarrow 10}
							{2*5 \Rightarrow 10}
					}
					{f\ 5\Rightarrow 10}
				\\
				\inferrule* [left=Op]
					{10-1 \Rightarrow 9}
					{10-1 \Rightarrow 9}
			}
			{g\ (f\ 5)\Rightarrow 9}
	}
  {\texttt{let x = 5 in g (f x)} \Rightarrow 9}
\]

\small
\[
\inferrule* [left=LD]
  {
		\inferrule* [left=App]
			{
				\texttt{fun a -> a+1} \Rightarrow \texttt{fun a -> a+1}
				\\
				\inferrule* [left=Op]
					{2+1 \Rightarrow 3}
					{2+1 \Rightarrow 3}
			}
			{\texttt{(fun a -> a+1) 2}\Rightarrow 3}
	}
  {\texttt{let fa = fun a -> a+1 in fa 2} \Rightarrow 3}
\]


\end{document}
